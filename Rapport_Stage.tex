\documentclass{polytech/polytech}

\typereport{stagedi5}

\reportyear{2017-2018}
\title{Réalisation d'un outil de cartographie sur le progiciel Amplitude}
\student{Romain}{ROUSSEAU}{romain.rousseau@etu.univ-tours.fr}
\academicsupervisor{Yannick}{KERGOSIEN}{yannick.kergosien@univ-tours.fr}
\industrialsupervisor[Responsable Cellule Architecture]{Alexandre}{DURAND}{alexandre.durand@soprabanking.com}

\company[images/logoSopra]{Sopra Banking}{47 rue Christiaan Huygens \\ 37073 Tours Cedex 2 - France}{www.soprabanking.com}

\resume{}

\motcle{}

\abstract{}

\keyword{}


%%%%%%%%%%%%%%%%%%%%%%%%%%%%%%%%%%%%%%%%%%%%%%%%%%%%%%%%%%%%%%%%%%%%%%%%%%%%%%%%%%%%%%%%%%%%%%%%%%%
%%%%%%%%%%%%%%%%%%%%%%%%%%%%%%%%%%%%%%%%%%%%%%%%%%%%%%%%%%%%%%%%%%%%%%%%%%%%%%%%%%%%%%%%%%%%%%%%%%%

\begin{document}

\chapter*{Introduction}


Ce rapport présente mon stage d'assistant ingénieur réalisé dans l'entreprise \textit{Sopra Banking Software} à Tours. Le sujet du stage était la réalisation d'un outil de cartographie pour le progiciel nommé \textit{Amplitude} de l'entreprise. Le stage a débuté le 9 avril 2018 avant de s'achever le 31 août de la même année.

Le stage de cinquième année a pour but d'appliquer les connaissances acquises lors des précédentes années d'étude. Il doit représenter une synthèse de l'ensemble des caractéristiques qu'un ingénieur doit avoir dans un environnement concret. Alors que le stage de troisième année sert à découvrir le monde de l'entreprise et que celui de quatrième année sert à faire le premier pas vers le métier d'ingénieur, le stage de dernière année permet de travailler dans une équipe installée pour mener à bien un projet, dans son intégralité ou tout du moins sa majorité selon les cas. Ce stage est censé donner une autonomie supérieure aux étudiants, qui doivent prendre des décisions et avoir la possibilité de donner leur avis sur les opérations en cours, ce qui en fait ainsi, une expérience nécessaire pour le métier. 

Il se déroule sur une période d'au minimum 18 semaines, ce qui permet de pouvoir réaliser des projets avec une envergure dont les étudiants-ingénieurs n'ont jamais eu l'occasion de faire durant leur cursus. Il s'agit du stage le plus important de la formation car il s'agit du plus complet, du plus formateur et de manière générale, du plus intéressant. Il apporte une expérience non négligeable et représente la dernière marche avec l'arrivée dans la vie active.

Le secteur de l'informatique est très porteur et les offres de stage sont donc très nombreuses et variées. L'opportunité de stage dans l'entre \textit{Sopra Banking Software} est venue lors du Forum des Entreprises qui se déroule à Polytech pendant le mois de novembre. J'y ai pu rencontrer de nombreuses entreprises et obtenir des entretiens avec plusieurs d'entre elles, dont notamment deux à Sopra sur le site de Tours. À la suite de ces entretiens, j'ai pu faire mon choix parmi les propositions. J'ai retenu celle de la cellule architecture de \textit{Sopra Banking Software} pour plusieurs raisons. Tout d'abord, le domaine de la banque m'a toujours intéressé, j'ai déjà fait mon stage de 4ème année chez Worldline du côté des paiements en ligne et je voulais découvrir un autre aspect du domaine. Ensuite, j'avais envie de faire partie d'un projet que ne soit pas monotone et qui me permette d'acquérir de nouvelles connaissances sur des technologies que je ne maîtrise pas ou très peu. Et c'était le cas avec cette offre qui combinait tous les aspects d'un projet, des prémices jusqu'au développement en passant par la modélisation ou encore le chiffrage, le tout en utilisant des technologies variées. 



\part{Présentation de l'entreprise et du projet}


\chapter{Sopra Steria et Sopra Banking Software}


\chapter{Le progiciel Amplitude}


\chapter{Présentation du projet}


\part{Tutoriels et formations}


\chapter{Maven}


\chapter{Spring Boot et Spring Data}


\chapter{Angular}


\part{Modélisation de l'outil et chiffrage du projet}


\part{Développement}


\chapter{title}

\chapter{Bilan du stage}


\appendix

\end{document}
